%
%	Configure LaTeX to produce a PDF "book" using the memoir class
%

\documentclass[10pt,oneside]{memoir}

%
%	Generic Configuration for memoir-based documents
%

\usepackage{layouts}[2001/04/29]


% In case we need a glossary, or index
\usepackage{glossaries}
\glstoctrue
\makeglossaries
\makeindex


% Basic page layout configuration
\def\mychapterstyle{default}
\def\mypagestyle{headings}


% Use 8.5 x 11 inch page layout
%
%	8.5 x 11 layout for memoir-based documents
%


%%% need more space for ToC page numbers
%\setpnumwidth{2.55em}
%\setrmarg{3.55em}

%%% need more space for ToC section numbers
%\cftsetindents{part}{0em}{3em}
%\cftsetindents{chapter}{0em}{3em}
%\cftsetindents{section}{3em}{3em}
%\cftsetindents{subsection}{4.5em}{3.9em}
%\cftsetindents{subsubsection}{8.4em}{4.8em}
%\cftsetindents{paragraph}{10.7em}{5.7em}
%\cftsetindents{subparagraph}{12.7em}{6.7em}

%%% need more space for LoF numbers
%\cftsetindents{figure}{0em}{3.0em}

%%% and do the same for the LoT
%\cftsetindents{table}{0em}{3.0em}

%%% set up the page layout
%\settrimmedsize{\stockheight}{\stockwidth}{*}	% Use entire page
%\settrims{0pt}{0pt}

%\setlrmarginsandblock{1.5in}{1.5in}{*}
%\setulmarginsandblock{1.5in}{1.5in}{*}

%\setmarginnotes{17pt}{51pt}{\onelineskip}
%\setheadfoot{\onelineskip}{2\onelineskip}
%\setheaderspaces{*}{2\onelineskip}{*}
%\checkandfixthelayout


% Use default packages for memoir setup
%
%	Default packages for memoir documents created by MultiMarkdown
%

\usepackage{fancyvrb}			% Allow \verbatim et al. in footnotes
\usepackage{graphicx}			% To enable including graphics in pdf's
\usepackage{booktabs}			% Better tables
\usepackage{tabulary}			% Support longer table cells
\usepackage[utf8]{inputenc}		% For UTF-8 support
\usepackage[T1]{fontenc}		% Use T1 font encoding for accented characters
\usepackage{xcolor}				% Allow for color (annotations)
\usepackage[sort&compress]{natbib} % Better bibliography support



% Configure default metadata to avoid errors
%
%	Configure default metadata in case it's missing to avoid errors
%

\def\myauthor{Author}
\def\defaultemail{}
\def\defaultposition{}
\def\defaultdepartment{}
\def\defaultaddress{}
\def\defaultphone{}
\def\defaultfax{}
\def\defaultweb{}


\def\mytitle{Title}
\def\subtitle{}
\def\keywords{}


\def\bibliostyle{plain}
% \def\bibliocommand{}

\def\myrecipient{}

% Overwrite with your own if desired
%\input{ftp-metadata}






\def\mytitle{Appellate Brief}
\def\myauthor{Matt Henry}
\def\bibliocommand{\bibliography{rrwa-brief}}
\def\bibliostyle{apacite}
\documentclass[12pt]{article}

% Emulate MS Word
\usepackage{wordlike}

% One inch margins
\PassOptionsToPackage{margin=1in}{geometry}

% Remove footnote indentation
\usepackage[hang,flushmargin]{footmisc} 

% Add extra blank line between paragraphs, and remove paragraph indentation
\usepackage[parfill]{parskip}
\parskip = 2\baselineskip

% Double spacing
\usepackage{setspace}
\setstretch{2}

% Header
\usepackage{fancyhdr}
\pagestyle{fancy}
\renewcommand{\headrulewidth}{0pt}
\setlength{\headheight}{57pt}
\lhead{Header line 1\\Line 2\\Line 3\\Line 4}

% Add space between text and footnotes section
\setlength{\skip\footins}{0.5in}

% Add a space before the footnote mark
\let\myfootnote\footnote
\renewcommand{\footnote}[1]{\myfootnote{~#1}}

% Don't justify along the right margin
\raggedright
\def\latexmode{memoir}
%
%	Get ready for the actual document
%

\usepackage[
	plainpages=false,
	pdfpagelabels,
	pdftitle={\mytitle},
	pagebackref,
	pdfauthor={\myauthor},
	pdfkeywords={\keywords}
	]{hyperref}
\usepackage{memhfixc}


%
%	Configure information from metadata for use in title
%

\ifx\latexauthor\undefined
\else
	\def\myauthor{\latexauthor}
\fi

\ifx\subtitle\undefined
\else
	\addtodef{\mytitle}{}{ \\ \subtitle}
\fi

\ifx\affiliation\undefined
\else
	\addtodef{\myauthor}{}{ \\ \affiliation}
\fi

\ifx\address\undefined
\else
	\addtodef{\myauthor}{}{ \\ \address}
\fi

\ifx\phone\undefined
\else
	\addtodef{\myauthor}{}{ \\ \phone}
\fi

\ifx\email\undefined
\else
	\addtodef{\myauthor}{}{ \\ \email}
\fi

\ifx\event\undefined
\else
	\date[\mydate]{\today}
\fi
\title{\mytitle}
\author{\myauthor}

\ifx\mydate\undefined
\else
	\date{\mydate}
\fi


\ifx\theme\undefined
\else
	\usetheme{\theme}
\fi

\begin{document}

\VerbatimFootnotes


\chapterstyle{\mychapterstyle}
\pagestyle{\mypagestyle}

% Frontmatter
\frontmatter

% Title Page
\maketitle
\clearpage

%
% Copyright Page
%

\vspace*{\fill}
\setlength{\parindent}{0pt}

\ifx\mycopyright\undefined
\else
	\textcopyright{} \mycopyright
\fi

\begin{center}
	\framebox{ \parbox[t]{1.5in}{\centering Formatted for \LaTeX \\ 
	by MultiMarkdown}}
\end{center}

\setlength{\parindent}{1em}
\clearpage

\tableofcontents
%\listoffigures
%\listoftables


\mainmatter


\part{Statement of the Issues Presented}
\label{statementoftheissuespresented}

Whether a reasonable employer would believe that an employee has a serious medical condition for the purposes of  \citet{29-cfr-825-303} when the employee asks for an extended period of time off.
I certify that no unauthorized assistance has been received or given in the completion of this work
Whether less than 24 hours notice is sufficient for a request for unforeseeable FMLA leave under 29 C.F.R. § 825.303 when the employee knew sooner than that that she was unable to work.

\part{Statement of the Case}
\label{statementofthecase}

Ms. Park alleges that Drummond Bioprocesses, Inc. (``Drummond'') interfered with her right to take medical leave under the Family Medical Leave Act (FMLA). She contends her communication with her supervisor was sufficient to notify Drummond of her intent to take such leave. Ms. Park further contends that her supervisor's response to her communication amounted to a grant of FMLA leave.

The district court agreed with Drummond that Ms. Park failed to prove that she gave notice of her intent to take FMLA leave. Since there is no dispute as to the facts on record, and those facts are insufficient to show Ms. Park gave notice, the court below determined that Drummond was entitled to summary judgment on the matter.

\part{Statement of Facts}
\label{statementoffacts}

Ms. Park was a senior lab technician at Drummond Bioprocesses, and a valued member of their research staff. R. at 8. Drummond researches medicines for illnesses that often attack patients who are already sick with cystic fibrosis. \emph{Id.} at 7. It was at a crucial moment in an important clinical trial of one of these treatments that Ms. Park stopped coming to work. R. at 14; \emph{see also} R. at 10 (on the importance of the stage of the trial). In so doing, she set the trial back a whole year. R. at 10. This setback jeopardized the funding of the trial. \emph{Id.} It also delayed the availability of an important treatment that would have lengthened lives and improved the quality of life for some very sick people. \emph{Id.}

When her brother died, Ms. Park took a week of bereavement leave. \emph{Id.} at 14. While she was away, the team struggled to follow the rigorous experimental protocol. \emph{Id.} at 10. During her leave, Ms. Park also struggled, though about whether she would be able to perform her duties on her return. \emph{Id.} When she returned, Ms. Park had trouble concentrating on even the minor tasks she was performing during her first three days back. \emph{Id.} When she resumed her normal, more rigorous, duties at the start of the following week, she had an even harder time, suffering flashbacks and insomnia. \emph{Id.} As one might expect, given the focus this work demanded and her trouble the prior week, Ms. Park was unable to giver her work the attention it demanded. \emph{Id.}

Despite Ms. Park's, apprehension about returning to work, her struggle to focus even while on light duty during her first few days back, and her admitted failure to adequately conduct herself during the first day of more rigorous work, she did not communicate any of this difficulty to anyone until after the close of business on the following day, when she emailed her supervisor, Dr. Emerson. \emph{Id.} In this email, she told him some of what she was going through and that she was not coming to work for a few weeks. \emph{Id.} at 18. Dr. Emerson replied the following morning that he understood that she was going through a difficult time, asked if she had a therapist to talk to, and asked her to ``[k]eep [him] posted.'' \emph{Id.} Over 36 hours later, Ms. Park finally replied to Dr. Emerson to tell him that she had a doctor's appointment coming up in five days. \emph{Id.} She did not contact anyone from Drummond during those five days, nor did she follow up after her appointment. \emph{Id.} at 15. When three days had gone by after the appointment, Drummond had to let Ms. Park go in order to salvage the project and its funding. \emph{Id.} at 10.

\part{Summary of the Argument}
\label{summaryoftheargument}

Drummond contends that Ms. Park's communication was insufficient in content and inadequately timed, and so did not constitute valid notice of intent to take FMLA leave. Because of this, the district court's grant of summary judgment should be affirmed.

An employer only has the burden to gather information on whether an employee's request for time off qualifies for FMLA leave when that request would make a reasonable employer think the employee had a serious medical condition. \emph{Brenneman v. MedCentral Health Sys.}, 366 F.3d 412, 421 (6th Cir. 2004). The Court of Appeals for the Sixth Circuit has said that an employee who communicates severe, debilitating psychological symptoms to her employer has not necessarily given notice of a serious medical condition, even when she has said that those symptoms prevent her from doing her job. \emph{Hammon v. DHL Airways, Inc.}, 165 F.3d 441, 450 (6th Cir. 1999). Given the direct applicability of this standard to the present case, this Court should find that Ms. Park's communication with her supervisor was not sufficient to trigger a duty to get more information about her FMLA status.

Not only was Ms. Park's notice insufficient in its content, but it was also inadequately timed to allow her to qualify for FMLA leave. The regulatory standard for timing is that notice must be ``as soon as practicable.'' 29 C.F.R. § 825.303 (2009). Since Ms. Park's condition was non-acute, and since she knew even before returning to work that she might be unable to do her job, last minute notice was not practicable. Further, examining the Department of Labor's intent in drafting that regulation shows that behavior like Ms. Park's was precisely what it was meant to prohibit.

Since Ms. Park's email to her supervisor did not indicate that she had a serious medical condition, and because that email was not adequately timed, she did not give sufficient notice of her intent to take FMLA-qualifying leave. Thus, this Court should affirm the below grant of summary judgment.

\part{Argument}
\label{argument}

\chapter{Because Ms. Park's Notice of Her Intent to Take FMLA-Qualifying Leave Was Both Insufficient in its Content and Inadequately Timed, This Court Should Affirm the Below Grant of Summary Judgement}
\label{becausems.parksnoticeofherintenttotakefmla-qualifyingleavewasbothinsufficientinitscontentandinadequatelytimedthiscourtshouldaffirmthebelowgrantofsummaryjudgement}

\section{The Content of Park's Email to Desmond Was Not Sufficient to Make a Reasonable Employer Understand That She Had a Serious Health Condition and Was Therefore Not Adequate to Create a Duty to Seek Further Information.}
\label{thecontentofparksemailtodesmondwasnotsufficienttomakeareasonableemployerunderstandthatshehadaserioushealthconditionandwasthereforenotadequatetocreateadutytoseekfurtherinformation.}

The default rule regarding notice of intent to take FMLA-qualifying leave is that an employer does not have a duty to seek more information regarding an employee's need for accommodation. \emph{Hammon} at 450. However, such a duty is created when the ``employer receives sufficient notice.'' \emph{Brenneman} at 422. A leading case in the Circuit on the sufficiency of notice held that ``the critical test for substantively-sufficient notice is whether the information that the employee conveyed to the employer was reasonably adequate to apprise the employer of the employee's request to take leave for a serious health condition that rendered him unable to perform his job.'' \emph{Id.} at 421. While the FMLA is silent on what constitutes a serious health condition, the implementing regulations provide some clarity: ``[f]or purposes of FMLA, `serious health condition' {\ldots} means an illness, injury, impairment or physical or mental condition that involves inpatient care {\ldots} or continuing treatment by a health care provider.'' 29 C.F.R. § 825.113 (2013). Case law also indicates that the bar is at least somewhat high, since an employee must do more than call in ``sick'' to indicate that she has a serious medical condition. \emph{Walton v. Ford Motor Co.}, 424 F.3d 481, 487 (6th Cir. 2005).

In \emph{Hammon} the Court of Appeals for the Sixth Circuit found that an employee who described his psychological symptoms to his employer and expressed doubts about his ability to perform his job had not given enough information to make his employer believe that he had a serious medical condition. \emph{Hammon} at 451. The plaintiff in that case was a Pilot who developed an anxiety disorder. \emph{Id} at 446. He had multiple conversations with his supervisors in which he complained of his symptoms and that he was considering resigning since he felt like he could no longer do the job. \emph{Id.} The supervisor told him to take a couple of days off to get his confidence back. \emph{Id.} The court did not think those conversations contained enough information for the employer to reasonably conclude that the plaintiff's obvious emotional trouble constituted a serious medical condition. \emph{Id.} The fact that the plaintiff did actually have a serious medical condition was irrelevant since he did not provide medical certification until after he had resigned. \emph{Id} at 451.

This decision is directly applicable to the present case. Ms. Park had severe psychological symptoms. Her symptoms were such that she did not feel as if she would be able to continue her work, and she told all of this to Dr. Emerson in her email. Indeed, if anything, what she communicated to Dr. Emerson was less severe than what the \emph{Hammon} plaintiff told his supervisor, since he did not think he would ever be well enough to do his job—Ms. Park felt she only needed a few weeks. Indeed, just like \emph{Hammon}'s plaintiff, Ms. Park's condition would likely have entitled her to FMLA leave if she had given better notice in her initial communication or provided medical certification before she was terminated.

The only major difference between the \emph{Hammon} facts and those in the present case is that Ms. Park mentioned that she planned to see a doctor. While there is no controlling authority on the question of whether mentioning seeing a doctor is enough to indicate that an employee has a serious medical condition, at least one court in this circuit applying \emph{Brenneman} and \emph{Walton} held that stating intent to see a doctor is not sufficient to turn a request for time off into notice of a serious medical condition. \emph{O'Connor v. Busch's Inc.}, 07--11090, 2008 WL 3913688 (E.D. Mich. Aug. 19, 2008). Indeed, given that Ms. Park had suffered such a horrible personal tragedy, one would only expect her to go see a therapist. Since she did not communicate any of the trouble she had since returning until she said she was unable to come in, and since she did not even state that she planned to see a doctor until Dr. Emerson asked, this Court ought not to take that statement as indicating that she had a serious medical condition.

\chapter{Even if the Content of Ms. Park's Notice Was Sufficient in its Content, its Timing Was Inadequate for a Condition That Had Been Ongoing for at Least a Week Prior to Her Telling Anyone.}
\label{evenifthecontentofms.parksnoticewassufficientinitscontentitstimingwasinadequateforaconditionthathadbeenongoingforatleastaweekpriortohertellinganyone.}

\section{Ms. Park's Notice Was not ``Practicable''}
\label{ms.parksnoticewasnotpracticable}

The standard for timing of notice for unforeseeable FMLA leave is ``as soon as practicable.'' 29 C.F.R. § 825.303(a) (2012). The Court of Appeals for the Sixth Circuit elaborated on that standard, saying: “[w]hat is practicable, both in terms of the timing of the notice and its content, will depend upon the facts and circumstances of each individual case.” \emph{Cavin v. Honda of Am. Mfg., Inc.}, 346 F.3d 713, 724 (2003, quoting \emph{Manuel v. Westlake Polymers Corp.}, 66 F.3d 758, 764 (5th Cir.1995)). The standard must thus be applied fluidly, given the exigencies of particular circumstances.

The circumstances in \emph{Cavin} were as unforeseeable and emergent as one could imagine since the plaintiff in that case was injured in a motorcycle accident. \emph{Cavin} at 716. In that case, same-day notice was adequate given the acute nature of the injury. \emph{Id} at 723. Applying the practicableness standard in that case, the Court of Appeals points out that rigid timing requirements for unforeseeable conditions run counter to the intent of the implementing regulation. \emph{Id} at 726. Ultimately, the circumstances of the case are critical to the court's decision.

There is a stark contrast between the sudden, emergent scenario in \emph{Cavin} and the weeks-long, accumulating nature of Ms. Park's injury. Whereas it would have been literally impossible for the \emph{Cavin} plaintiff to have given notice sooner, since there was nothing to give notice of before his accident, Ms. Park began to suffer weeks before. Her injury occurred when her brother was murdered. 

While she was on bereavement leave, no one, including Ms. Park, should have been concerned that her psychological and physical pain were anything other than normal under the tragic circumstances—that is the purpose of bereavement leave. However, Ms. Park should have known and did at least strongly suspect that she would be unable to work when her symptoms, as it may be fair to call them at this point, did not abate by the time she was scheduled to return. Nonetheless, she did not communicate her difficulties to anyone on her team.

In the first three days after Ms. Park came back after her bereavement leave, she was already having trouble concentrating despite the fact that the work she was doing at that point was not very hard compared to the demands of the trial regimen. Since no one would know the rigors of the experimental protocol better than she, she should have known at that point that she would be unable to do the much more demanding work that would be starting back up the following week. Indeed, she continued to have these very doubts, but kept them to herself.

If she had found after the first Monday back, or even the following morning (given than that insomnia was part of the problem), that she could not work, that may have been enough time to let the team come up with another staffing solution. Yet despite all of her difficulty, she did not communicate with her supervisor or anyone else at either of those times to let them know she was having trouble so they could start figuring out a plan for staffing.

Contrast this long, painful story with the much less complicated one in \emph{Cavin}. Surely in that case, no notice was practicable until the injury occurred. In the present case, it is difficult to pinpoint the precise moment at which Ms. Park's awareness of the gravity of her condition made her certain she would be unable to to work. Nonetheless, the record shows several points at which 1) notice would have been practicable, in that Drummond could have conceivably come up with an alternate plan for staffing trial; and 2) Ms. Park knew she would be unable to keep up with the demands of her work. Thus, whatever notice would have been practicable would have to have been given some time before the notice Ms. Park actually gave.

\section{The Department of Labor Intended to Prevent Last-Minute Notice of Unforeseeable Leave}
\label{thedepartmentoflaborintendedtopreventlast-minutenoticeofunforeseeableleave}

The fluidity required in applying this standard has only increased since \emph{Cavin} was decided. The regulation the \emph{Cavin} court was interpreting qualified the practicableness requirement as ``within no more than one or two working days of learning of the need for leave.'' 29 C.F.R. § 825.303 (2009). After \emph{Cavin} was decided, the Department of Labor drafted a new version of the regulation that retained the practicableness standard, but replaced the more specific number-of-days gloss with a less concrete requirement that the employee give notice ``promptly.'' 29 C.F.R. § 825.303(a) (2012). Explaining the change, the Department of Labor said that what was prompt would depend both on the industry and on the practices of the particular workplace. 73 FR 67934, 68007 (2009). The Department explicitly assumed that reliance on those practices would prevent last-minute notice that ``is a particular problem in \emph{time-sensitive, critical services, and interdependent} jobs.'' \emph{Id} (emphasis added).

Since Ms. Park gave her notice well after close of business on the last day before she intended to come in, she could not have expected anyone to see her email until first-thing the following day. Such notice was effectively the kind of last-minute warning the regulation was intended to prevent. Further, the work Drummond was depending on Ms. Park to do was: 1) extremely time-sensitive; 2) critical, in that lives and taxpayer dollars were at stake; and 3) highly interdependent, since she and her only other equally qualified co-worker alternated shifts. Since Ms. Park was supposed to supposed to inject the mice when she got to the laboratory on the first day she did not come in, Callie Green would not have been expecting to come in for 14 hours after that time, and so was likely unavailable. If the \emph{Cavin} interpretation of the practicableness standard left any ambiguity as to whether Ms. Park's notice was adequately timed, the fact that situations like this are precisely what the Department of Labor meant to discourage, resolves that ambiguity in Drummond's favor. 

\part{Conclusion}
\label{conclusion}

Ms. Park's notice was not sufficient to entitle her to FMLA leave. It was not adequate to make her employer think she had a serious health condition. Even if it was adequate with regard to its content, the timing of the notice was not practicable. As a result, taxpayer money was wasted and very sick people were denied treatments that could have lengthened and improved their lives. While the tragedy she suffered was undeniably horrific, it does not absolve her of her responsibility to her employer, or to the people her work could have helped. 

Since Ms. Park has again failed to prove she gave notice of her intent to take FMLA leave, Drummond requests that the Court affirm the below grant of summary judgment.

%
%	MultiMarkdown default footer file
%


% Back Matter
\if@mainmatter
	we're in main
	\backmatter
\fi


% Bibliography

\ifx\bibliocommand\undefined
\else
	\bibliographystyle{\bibliostyle}
	\bibliocommand
\fi



% Glossary
\printglossaries


% Index
\printindex



\end{document}
